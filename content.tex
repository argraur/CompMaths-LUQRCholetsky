\section{Задача}

\begin{enumerate}
	\item Задать положительную определенную действительную и симметричную матрицу A\([100,100]\)
	\item С помощью функции \([S,J]=eig(A)\) найти собственные числа и собственные вектора матрицы А
	\item Меняем в цикле первое собственное число \(J(1,1)=J(1,1)*10^i, i = 1..15\). Генерируем матрицы с заданным числом обусловленности \(A=S*J*S^{-1}\).
	\item Для каждой из полученных матриц А определяем погрешность определения вектора Х, для чего : задаем вектор Х, вычисляем вектор \(В=А*Х\), выполняем LU разложение \([L,U]=lu(A)\), находим решение системы линейных алгебраических уравнений \(L*U*X_1=B\) : \(Y=L^{-1}*B\); \(X_1=U^{-1}*Y\), выполняем QR разложение, находим решение СЛАУ, а также выполняем решение методом Холецкого.
	\item Для каждого из полученных решений находим погрешность \(err(i)=\frac{norm(X-X_1)}{norm(X)}\)
	\item Строим и анализируем график погрешности в логарифмическом масштабе \(plot(log10(err))\).
\end{enumerate}

\newpage

\section{Ход выполнения работы}

В среде MATLAB был написан код, выполняющий поставленные задачи.

\subsection{Программный код для анализа}
\begin{code}
	\inputminted[breaklines=true, xleftmargin=1em, linenos, frame=single, framesep=10pt, fontsize=\footnotesize, firstline=1, lastline=33]{matlab}{listings/luqr.m}
	\caption{Код в среде MATLAB}
\end{code}

\section{Полученные графики}

\begin{figure}[H]
	\begin{center}
		\includegraphics[scale=0.6]{LUQRCholLog}
		\caption{Графики погрешностей в логарифмическом масштабе. LU - черным, QR - зеленым, метод Холецкого - красным.}
		\label{pic:luqr} % название для ссылок внутри кода
	\end{center}
\end{figure}

\begin{figure}[H]
	\begin{center}
		\includegraphics[scale=0.6]{LUQRChol}
		\caption{Графики погрешностей. LU - черным, QR - зеленым, метод Холецкого - красным.}
		\label{pic:luqr} % название для ссылок внутри кода
	\end{center}
\end{figure}


\section{Вывод}
В случае с положительной определенной действительной и симметричной матрицей, LU-разложение показывает наилучший результат с минимальной погрешностей, на втором месте QR-разложение, и с худшей погрешностью даёт результат метод Холецкого. LU-разложение подходит для систем с небольшим числом неизвестных, QR-разложение и метод разложения Холецкого более эффективны для систем с большим числом неизвестных, но метод Холецкого подходит только к определенному виду матриц. Лучшим выбором будет QR-разложение.